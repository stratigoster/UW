\documentclass[onecolumn,11pt]{article}
\setlength{\oddsidemargin}{0in}
\setlength{\evensidemargin}{0in}
\setlength{\textwidth}{6.5in}
\setlength{\textheight}{9in}
\setlength{\topmargin}{0in}
\setlength{\headheight}{0in}
\setlength{\headsep}{0in}
\usepackage{amsfonts}
\usepackage{alltt}
\usepackage[latin1]{inputenc}
\usepackage{moreverb}
\usepackage{graphicx}
\usepackage{latexsym}
\usepackage{amsmath}
\usepackage{fullpage}
\usepackage{alltt}

\title{CS442 Assignment 4}
\author{Nissan Pow\\20187246\\npow}
\date{\today}

\newtheorem{definition}{Definition}
\newtheorem{theorem}{Theorem}
\newcommand{\nonterm}[1]{\ensuremath{\langle\mbox{\emph{#1}}\rangle}}
\newcommand{\post}[2]{\ensuremath{\frac{\begin{array}{c}#1\end{array}}{\begin{array}{c}#2\end{array}}}}
\newcommand{\sem}[1]{\ensuremath{[\![#1]\!]}}

\begin{document}
\maketitle

\section*{Part A}
\begin{enumerate}
  \item{Yes there is a type that is a subtype of every other type - it is called Bottom. There is a function type that is a supertype of every other function type, and its type is Top $\to$ Bottom}
  \item{The rule says that every polymorphic type is a subtype of it's specialized self. This makes sense, since we can always use the polymorphic type wherever the specialized one is valid.}
  \item{}
  \item{}
  \item{}
\end{enumerate}

\section*{Parts B-E}
FILE: a5.e
\verbatimtabinput[2]{a5.e}

FILE: rule.e
\verbatimtabinput[2]{rule.e}

FILE: database.e
\verbatimtabinput[2]{database.e}

\clearpage

\subsection*{Transcript}
\verbatimtabinput[2]{transcript}

NOTES: \\
blah.pl is to test that the cut works correctly \\
t1.pl is the example from page 168 of the notes, and illustrates a non-terminating search \\
t2.pl is the example from page 167 of the notes, and just tests that searching works properly \\
t3.pl tests whether we can handle the cut as the last item in a rule. \\

\end{document}

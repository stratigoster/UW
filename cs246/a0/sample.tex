% Created by Adam Drenth, CS 246 Tutor, May 3 2006
%
% This file will give you a basic introduction to LaTeX since the easiest way to learn the
% language is by example.  This file can also be used as a template for building LaTeX files for your 
% assignments.
% 
% The basic structure of a LaTeX file is:
%   \documentclass{..}[..]
%   ..
%   package declarations
%   ..
%   tile information (if you wish to use it) like \title, \date, or \author.  To print the
%	title information you need to include \maketitle in the body of the document.
%   \begin {document}
%   ..
%   Your LaTeX code
%   ..
%   \end{document}
%
% Note: Pay careful attention to how LaTeX handles whitespace.  Most of the time it just 
%       ignores it.  Sometimes you may need to use a \\ to tell it to start a new line.
%
% Make sure that you at least read the first 10 pages of the Not So Short Introduction to LaTeX.  Its URL is: 
%
%    www.ctan.org/tex-archive/info/lshort/english/lshort.pdf
%
% It will give you a general feel for the language and a starting point for debugging any small errors
% you may have while creating a LaTeX document.

\documentclass{article}[12pt]		    % Sets the type of document you wish to create

% Load packages into your LaTeX file
\usepackage{epsfig}	% Importing .eps files
\usepackage{verbatim}   % Importing C++ code using \verbatiminput{file}	

% Setting the Title and Author
\title{Example \LaTeX\ File}
\author{Your Name}

% Since no date is specified, it will use the current date in the title

% Start body of document
\begin{document}
\maketitle		% Print your title

First notice 
how the white    space       is	    setup in the .tex file.  
\LaTeX\ will actually remove most of it, and 
make it appear 
as 2 sentences, one after the other, instead of broken up text. \\  % This forces a newline

Spaces serve only to seperate words and commands.  A double newline is interpreted as a new paragraph.

\tiny{You} \scriptsize{can} \small{set} \normalsize{fonts} \large{to} \Large{different} \LARGE{sizes}\normalsize{.}

% Start a section in your document.  The "*" makes it so the section title does not have a number
\section*{Special and Punctuation Characters}

% Note how quotation is done here.  It is 2 ` characters and 2 ' characters. Also notice that \ makes the special
% character just a character
There are several ``special'' characters: \\ \$, \&, \%, \#, \_, \{, \}  

The punctuation characters are: . : ; , ? ! ` ' ( ) [ ] - / * @

% This type of Section will have a number in front of it 
\section{Encapsulated PostScript and C++ Code}
Its \emph{important} to know how to display .eps {\textbf files}, .cc or .h files 
easily for the assignments in this course.
 
For example:

% center centers the picture on the page
\begin{center}
    Here is a UML diagram from xfig saved as Encapsulated PostScript
    
    % This is now you import a file from xfig  
    % Note: The file is in Encapsulated PostScript!
    \epsfig{file=Clone.eps, scale=0.5}
\end{center}

Here is some C++ code loaded from a .cc file
    % This is the easiest way to import C++ code into your assignments.  Since whitespace is normally
    % ignored, this beats the hell out of trying to hard code it into your document. There is another
    % way to do it with the verbatim package that allows you to maintain your indentation in your code,
    % but this means you need to copy and paste your code in order to compile and test it.
\verbatiminput{Customer.cc}

\pagebreak % Creates a page break

% Information on how to compile your LaTeX document into a pdf document
% Notice this it is subsection instead of section.  This title will be labelled 1.1
\subsection{Compiling \LaTeX}
The commands are as follows:

% This creates a list without bullets or numbers
\begin{description}
    \item {\tt vim sample.tex} \emph{ \% Create your \LaTeX\ file } 
    \item {\tt latex sample.tex} \emph{ \% Compile your \LaTeX\ file into a
    .dvi file } 
    \item {\tt dvi2pdf sample.dvi sample.pdf} \emph{ \% Convert to .pdf file } 
\end{description}

\section{Tips}  % Section 2

% Enumerate allows you to list things, but uses numbers instead of bullets
\begin{enumerate}
    \item Build your \LaTeX\ file gradually. My sure you compile at each step along the
    way to help you find errors early and quickly.
    \item Use the Internet!  There is a lot of information and sample code.  This is the best way to learn \LaTeX
\end{enumerate}

This is just the very basics of \LaTeX.  There is plenty of additional
information at:

% Itemize lists off items using bullets.  Notice that it has a \begin and \end.
\begin{itemize}
    \item www.ctan.org/tex-archive/info/lshort/english/lshort.pdf
    \item http://ist.uwaterloo.ca/cs/latex
    \item master.stat.tku.edu.tw/docu/latex/old/figsinltx.pdf
    \item http://www.cs.nyu.edu/~yap/student/LatexBasics.html
\end{itemize}

\end{document}	% End of body of document
